\documentclass[10pt]{article}
\usepackage[a4paper, margin=0.5in]{geometry}
\usepackage{amsmath, amssymb, enumitem, multicol}
\usepackage[table]{xcolor}

\setlist{nolistsep} % Removes spacing between list items to save space

\title{\vspace{-4ex}MXB103 Exam Notes\vspace{-1ex}}
\author{}
\date{}

\begin{document}
\maketitle
\thispagestyle{empty} % Remove page number

\begin{multicols}{2} % Use two columns to fit more information

\section*{Ordinary Differential Equations}
\subsection*{Euler's Method}
what's given: function, where you want to approximate, step value (h), and initial condition (when n = 0)\\
follow the formula: $ y_{n+1} = y_{n} + hf'(x_{n}) $ \\
\begin{tabular}{ c c c }
    n & $ x_{n} $ & $ y_{n} $ \\
    0 & initial x & initial y \\ 
    1 & initial x + h & follow fomula\\
    2 & $ x_{1} + h $ & follow fomula\\
    3 & $ x_{2} + h $ & follow fomula\\
    4 & $ x_{3} + h $ & follow fomula\\
    5 & $ x_{4} + h $ & follow fomula\\
\end{tabular}

\subsection*{Second Order Taylor Method}

\begin{enumerate}
    \item Identify \( f(x) \) and the point \( a \).
    \item Compute derivatives \( f'(x) \) and \( f''(x) \).
    \item Evaluate \( f'(a) \) and \( f''(a) \).
    \item Construct the Taylor polynomial:
    \[ P_2(x) = f(a) + f'(a)(x-a) + \frac{f''(a)}{2}(x-a)^2 \]
    \item Use \( P_2(x) \) to approximate \( f(x) \) near \( a \).
\end{enumerate}



\subsection*{Modified Euler's Method}

\begin{enumerate}
    \item Given \( y_n \) and \( x_n \), compute the initial slope \( k_1 = f(x_n, y_n) \).
    \item Predict the next value \( \tilde{y}_{n+1} = y_n + k_1 h \), where \( h \) is the step size.
    \item Compute the corrected slope \( k_2 = f(x_n + h, \tilde{y}_{n+1}) \).
    \item Calculate the next value \( y_{n+1} = y_n + \frac{h}{2}(k_1 + k_2) \).
    \item Update \( x_{n+1} = x_n + h \) and repeat from step 1 for further iterations.
\end{enumerate}

\subsection*{Runge-Kutta 4th Order (RK4)}

\begin{enumerate}
    \item Start with initial conditions \( y_n \) and \( x_n \).
    \item Calculate the first slope \( k_1 = h f(x_n, y_n) \).
    \item Compute the second slope \( k_2 = h f(x_n + \frac{h}{2}, y_n + \frac{k_1}{2}) \).
    \item Calculate the third slope \( k_3 = h f(x_n + \frac{h}{2}, y_n + \frac{k_2}{2}) \).
    \item Compute the fourth slope \( k_4 = h f(x_n + h, y_n + k_3) \).
    \item Estimate the next value \( y_{n+1} = y_n + \frac{1}{6}(k_1 + 2k_2 + 2k_3 + k_4) \).
    \item Increment \( x_{n+1} = x_n + h \) and use the new \( x_{n+1} \) and \( y_{n+1} \) for the next iteration.
\end{enumerate}

\columnbreak

\section*{Polynomial Interpolation}

\subsection*{Lagrange Interpolating Polynomial}
\begin{enumerate}
    \item sub abscissas into given function to get y values.
    \item find the Polynomial for the amount of abscissas. Ex: $ (x_{0}, y_{0}), (x_{1},y_{1}), (x_{2},y_{2}) $\\ $ \frac{(x - x_{1})(x-x_{2})}{(x_{0}-x_{1})(x_{0}-x_{2})} \cdot y_{0} $ + same thing for $ y_{1} $ and $ y_{2} $
\end{enumerate}


\subsection*{Newton's Divided Difference}
$ a_{0} = y_{0}$, $a_{1} = \frac{y_{1}-y_{0}}{x_{1}-x_{0}}$, $ a_{2} = \frac{\frac{y_{2}-y_{1}}{x_{2}-x_{0}} - \frac{y_{1}-y_{0}}{x_{1}-x_{0}}}{x_{2}-x_{0}}$\\

\begin{tabular}{ c c c c c }
    $ x_{i} $ & \text{zeroth} & \text{first} & \text{second} & \text{third}\\
    $ x_{0} $ & \cellcolor{yellow!30} $ f[x_{0}] $ & \cellcolor{yellow!30} $ f[x_{0}, x_{1}] $ & \cellcolor{yellow!30} $ f[x_{0}, x_{1}, x_{2}] $ & \cellcolor{yellow!30} $ f[x_{0}, x_{1}, x_{2}, x_{3}] $\\
    $ x_{1} $ & $ f[x_{1}] $ & $ f[x_{1}, x_{2}] $ & $ f[x_{1}, x_{2}, x_{3}] $ \\
    $ x_{2} $ & $ f[x_{2}] $ & $ f[x_{2}, x_{3}] $ & \\
    $ x_{3} $ & $ f[x_{3}] $ & \\
\end{tabular} \\

$ P_{3}(x) = f[x_{0}] = f[x_{0}, x_{1}](x-x_{0}) + f[x_{0}, x_{1}, x_{2}](x-x_{0})(x-x_{1}) + f[x_{0}, x_{1}, x_{2}, x_{3}](x-x_{0})(x-x_{1})(x-x_{2})$\\
\begin{enumerate}
    \item Sub given x and y values into first and second columns of the table.
    \item Compute first divided difference using above $ a_{1} $ formula.
    \item Compute second divided difference using above $ a_{2} $ formula
    \item Sub into above formula.
\end{enumerate}


\subsection*{Newton's Forward Difference}
ONLY use when the abscissas are equally spaced.\\

\begin{tabular}{ c c c c c }
    $ x_{i} $ & $ y_{i} $ & $ \Delta y_{i} $ & $ \Delta^{2} y_{i} $ & $ \Delta^{3} y_{i} $ \\
    \hline
    $ x_{0} $ & $ y_{0} $ & $ \cellcolor{yellow!30} \Delta y_{0} $ & $ \cellcolor{yellow!30} \Delta^{2} y_{0} $  & $ \cellcolor{yellow!30} \Delta^{3} y_{0} $ \\
    $ x_{1} $ & $ y_{1} $ & $ \Delta y_{1} $ & $ \Delta^{2} y_{1} $ & \\
    $ x_{2} $ & $ y_{2} $ & $ \Delta y_{2} $ & & \\
    $ x_{3} $ & $ y_{3} $ & & & \\
\end{tabular}

\begin{enumerate}
    \item sub given x and y values into first and second columns of table.
    \item Compute remaining columns. Ex: $ \Delta y_{0} =  y_{1} - y_{0} $
    \item Sub into given polynomial equation.
\end{enumerate}


\end{multicols}

\pagebreak
\section*{Helpful Additional Info}

\subsection{Differential equation rules:}

\textbf{Constant Rule:} \( c' = 0 \) where \( c \) is a constant.

\textbf{Power Rule:} \( (x^n)' = nx^{n-1} \) for any real number \( n \).

\textbf{Sum Rule:} \( (u + v)' = u' + v' \)

\textbf{Difference Rule:} \( (u - v)' = u' - v' \)

\textbf{Product Rule:} \( (uv)' = u'v + uv' \)

\textbf{Quotient Rule:} \( \left(\frac{u}{v}\right)' = \frac{u'v - uv'}{v^2} \)

\textbf{Exponential Rule:} \( (a^{x})' = a^{x}\ln(a) \) where \( a > 0 \) and \( a \neq 1 \).

\textbf{Logarithmic Rule:} \( (\ln(x))' = \frac{1}{x} \) for \( x > 0 \).

\textbf{Trigonometric Rules:}

\( (\sin(x))' = \cos(x) \)

\( (\cos(x))' = -\sin(x) \)

\( (\tan(x))' = \sec^2(x) \)

\textbf{Inverse Trigonometric Rules:}

\( (\arcsin(x))' = \frac{1}{\sqrt{1-x^2}} \)

\( (\arccos(x))' = -\frac{1}{\sqrt{1-x^2}} \)

\( (\arctan(x))' = \frac{1}{1+x^2} \)


\subsection{Initial Value Problem Examples:}
$ Given: \frac{dy}{dx} = (6x - 3), y(0) = 4 $\\
$ dx \cdot \frac{dy}{dx} = (6x - 3) dx $ \\
$ dy = (6x - 3) dx $ \\
\(\int dy \,\) = \(\int (6x - 3) \,dx\) \\
$ y = \frac{6x^{2}}{2} - 3x + C $ \\
Sub in initial value \\
$ 4 = 3(0)^{2} - 3(0) + C $ \\
$ C = 4 $ \\
Plug constant C back into function:\\
$ y = \frac{6x^{2}}{2} - 3x + 4 $\\\\

\noindent $ Given: \frac{dy}{dx} = 2xy $, with initial value: $ y(0) = 3 $ \\
$ dy = 2xy \cdot dx $ \\
$ \frac{1}{y} \cdot dy = \frac{2xy}{y} dx $ \\
\(\int \frac{1}{y} dy \,\) = \(\int 2x \cdot dx \,\) \\
$ lny = \frac{2x^{2}}{2} + C $ \\
$ lny = x^{2} + C $ \\
$ e^{lny} = e^{x^{2}+C} $ \\
$ y = e^{C} \cdot e^{x^{2}} $ \\
$ y = Ce^{x^{2}} $ \\
Sub in initial value \\
$ 3 = Ce^{0} $ \\
$ C = 3 $


\end{document}
